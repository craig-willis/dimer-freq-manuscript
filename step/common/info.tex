
\documentclass[aip,jcp,showpacs,superscriptaddress,groupedaddress]{revtex4-1}  % for review and submission
%\documentclass[aip,jcp,preprint]{revtex4-1}
%\documentclass[journal=jctcce,manuscript=article]{achemso}


\usepackage{graphicx}  % needed for figures
\usepackage{dcolumn}   % needed for some tables
\usepackage{bm}        % for math
\usepackage{amssymb}   % for math
\usepackage{amsmath}
\usepackage{mathtools}
\usepackage{booktabs}
\usepackage{color,soul}
\usepackage{braket}
\usepackage{float}

%\usepackage{authblk}

% avoids incorrect hyphenation, added Nov/08 by SSR
\hyphenation{ALPGEN}
\hyphenation{EVTGEN}
\hyphenation{PYTHIA}

\begin{document}

\newcommand*{\citen}[1]{%
  \begingroup
    \romannumeral-`\x % remove space at the beginning of \setcitestyle
    \setcitestyle{numbers}%
    \cite{#1}%
  \endgroup   
}
\newcommand\Tstrut{\rule{0pt}{2.6ex}}         % = `top' strut
\newcommand\Bstrut{\rule[-1.5ex]{0pt}{0pt}}   % = `bottom' strut

% the following line is for submission, including submission to the arXiv!!
\hspace{5.2in} \mbox{CalStateLA-Pub-01/xxx-E}

\title{Supporting Information for: Anharmonic vibrational structure of the carbon dioxide dimer with a many-body potential energy surface.}

\author{Samuel \surname{Maystrovsky}}
\affiliation{Department of Chemistry, Biochemistry and Physics, The University of Tampa, 401 West Kennedy Boulevard, Tampa, Florida 33606, USA}
\author{Murat \surname{Ke\c{c}eli}}
\affiliation{Computational Science Division, Argonne National Laboratory, Lemont, Illinois, 60439}
\author{Olaseni \surname{Sode}}\email[The author to whom the correspondence should be addressed. Electronic mail: ] {osode@calstatela.edu}
\affiliation{Department of Chemistry, Biochemistry and Physics, The University of Tampa, 401 West Kennedy Boulevard, Tampa, Florida 33606, USA}
\affiliation{Department of Chemistry and Biochemistry, California State University, Los Angeles, 5151 State University Drive, Los Angeles, 90032, USA}

\date{\today}


\pacs{}

\maketitle

\section[S1]{Quartic Force Fields}
The harmonic, cubic, and quartic force constants for the quartic force field (QFF) used for the CO$_2$ monomer and the intramolecular motions of the CO$_2$ dimer are shown in Tables \ref{table:qff_monomer} and \ref{table:qff_dimer}. The derivatives were obtained numerically with the central difference approximation and a step size of $5.0 \times10^{-3}$ \AA. Due to the symmetry in the monomer and dimer, some of the force constants are equivalent, although they appear in the tables as unique values. 

\begin{table}[H]
\centering
\caption{The quartic force field of the CO$_2$ monomer determined with the \emph{mb}CO2 potential.}
\label{table:qff_monomer}
\begin{ruledtabular}
\begin{tabular}{ccc}
Force constant & Value        & Unit         \\
\hline \Tstrut
%$g_1$             & $-2.64E-07$    & $E_{\rm h}\ {\rm \AA}^{-1}\ {\rm u}^{-1/2}$ \\
$h_{22}=h_{\bar{2}\bar{2}}$        & $0.061211$  & $E_{\rm h}\ {\rm \AA}^{-2}\ {\rm u}^{-1}$   \\
$h_{11}$            & $0.24710$   & $E_{\rm h}\ {\rm \AA}^{-2}\ {\rm u}^{-1}$   \\
$h_{33}$            & $0.77413$  & $E_{\rm h}\ {\rm \AA}^{-2}\ {\rm u}^{-1}$   \\
$t_{111}$           & $0.30760$   & $E_{\rm h}\ {\rm \AA}^{-3}\ {\rm u}^{-3/2}$ \\
%$t_{333}$           & $8.02E-07$     & $E_{\rm h}\ {\rm \AA}^{-3}\ {\rm u}^{-3/2}$ \\
$u_{1111}$          & $0.29234$  & $E_{\rm h}\ {\rm \AA}^{-4}\ {\rm u}^{-2}$ \\
$u_{2222}=u_{\bar{2}\bar{2}\bar{2}\bar{2}}$    & $0.11415$  & $E_{\rm h}\ {\rm \AA}^{-4}\ {\rm u}^{-2}$ \\
$u_{3333}$          & $3.7009$  & $E_{\rm h}\ {\rm \AA}^{-4}\ {\rm u}^{-2}$    \\
$t_{122}=t_{1\bar{2}\bar{2}}$      & $-0.087258$ & $E_{\rm h}\ {\rm \AA}^{-3}\ {\rm u}^{-3/2}$ \\
%$t_{322}=t_{322}$      & $-4.66E-06$    & $E_{\rm h}\ {\rm \AA}^{-3}\ {\rm u}^{-3/2}$ \\
%$t_{311}$           & $-2.06E-06$    & $E_{\rm h}\ {\rm \AA}^{-3}\ {\rm u}^{-3/2}$ \\
$t_{133}$           & $1.0291$  & $E_{\rm h}\ {\rm \AA}^{-3}\ {\rm u}^{-3/2}$ \\
%$t_{311}$           & $-2.06E-06$    & $E_{\rm h}\ {\rm \AA}^{-3}\ {\rm u}^{-3/2}$ \\
$u_{1333}$          & $-0.0066063$ & $E_{\rm h}\ {\rm \AA}^{-4}\ {\rm u}^{-2}$    \\
$u_{3111}$          & $-0.0029963$ & $E_{\rm h}\ {\rm \AA}^{-4}\ {\rm u}^{-2}$    \\
$u_{1122}=u_{11\bar{2}\bar{2}}$    & $-0.17747$ & $E_{\rm h}\ {\rm \AA}^{-4}\ {\rm u}^{-2}$  \\
$u_{1133}$          & $1.0579$  & $E_{\rm h}\ {\rm \AA}^{-4}\ {\rm u}^{-2}$    \\
$u_{22\bar{2}\bar{2}}$          & $0.037833$  & $E_{\rm h}\ {\rm \AA}^{-4}\ {\rm u}^{-2}$    \\
$u_{2233}=u_{\bar{2}\bar{2}33}$    & $-0.72250$ & $E_{\rm h}\ {\rm \AA}^{-4}\ {\rm u}^{-2}$    \\
$u_{2213}=u_{\bar{2}\bar{2}13}$    & $-0.011696$ & $E_{\rm h}\ {\rm \AA}^{-4}\ {\rm u}^{-2}$   
\end{tabular}
\end{ruledtabular}
\end{table}

\begin{table}[H]
\centering
\caption{The quartic force field of the intramolecular vibrations of the CO$_2$ dimer determined with the \emph{mb}CO2 potential.}
\label{table:qff_dimer-1mr}
\begin{ruledtabular}
\begin{tabular}{cccccc}
Force constant & Value        & Unit  & Force constant & Value        & Unit     \\
\hline \Tstrut
?dimer-QFF-1MR?
\end{tabular}
\end{ruledtabular}
\end{table}

%\begin{table}[H]
%\centering
%\caption{The 2MR harmonic force constants of the intramolecular vibrations of the CO$_2$ dimer determined with the \emph{mb}CO2 potential.}
%\label{table:qff_dimer-2mr-h}
%\begin{ruledtabular}
%\begin{tabular}{cccccc}
%Force constant & Value      &  & Force constant & Value      &    \\
%%\begin{tabular}{ccc|ccc}
%%Force constant & Value        & Unit  & Force constant & Value        & Unit     \\
%\hline \Tstrut
%?dimer-QFF-2MR-H?
%\end{tabular}
%\end{ruledtabular}
%\end{table}

\begin{table}[H]
\centering
\caption{The 2MR symmetric quartic force constants of the intramolecular vibrations of the CO$_2$ dimer determined with the \emph{mb}CO2 potential.}
\label{table:qff_dimer-2mr-q1}
\begin{ruledtabular}
\begin{tabular}{cccccc}
Force constant & Value      &  & Force constant & Value      &    \\
%\begin{tabular}{ccc|ccc}
%Force constant & Value        & Unit  & Force constant & Value        & Unit     \\
\hline \Tstrut
?dimer-QFF-2MR-Q1?
\end{tabular}
\end{ruledtabular}
\end{table}

\begin{table}[H]
\centering
\caption{The 2MR cubic force constants of the intramolecular vibrations of the CO$_2$ dimer determined with the \emph{mb}CO2 potential.}
\label{table:qff_dimer-2mr-t}
\begin{ruledtabular}
\begin{tabular}{cccccc}
Force constant & Value      &  & Force constant & Value      &    \\
%\begin{tabular}{ccc|ccc}
%Force constant & Value        & Unit  & Force constant & Value        & Unit     \\
\hline \Tstrut
?dimer-QFF-2MR-C?
\end{tabular}
\end{ruledtabular}
\end{table}

\begin{table}[H]
\centering
\caption{The 2MR asymmetric quartic force constants of the intramolecular vibrations of the CO$_2$ dimer determined with the \emph{mb}CO2 potential.}
\label{table:qff_dimer-2mr-q2}
\begin{ruledtabular}
\begin{tabular}{cccccc}
Force constant & Value      &  & Force constant & Value      &    \\
%\begin{tabular}{ccc|ccc}
%Force constant & Value        & Unit  & Force constant & Value        & Unit     \\
\hline \Tstrut
?dimer-QFF-2MR-Q2?
\end{tabular}
\end{ruledtabular}
\end{table}

\begin{table}[H]
\centering
\caption{The 3MR cubic force constants of the intramolecular vibrations of the CO$_2$ dimer determined with the \emph{mb}CO2 potential.}
\label{table:qff_dimer-3mr-c}
\begin{ruledtabular}
\begin{tabular}{cccccc}
Force constant & Value      &  & Force constant & Value      &    \\
%\begin{tabular}{ccc|ccc}
%Force constant & Value        & Unit  & Force constant & Value        & Unit     \\
\hline \Tstrut
?dimer-QFF-3MR-C?
\end{tabular}
\end{ruledtabular}
\end{table}

\begin{table}[H]
\centering
\caption{The 3MR quartic force constants of the intramolecular vibrations of the CO$_2$ dimer determined with the \emph{mb}CO2 potential.}
\label{table:qff_dimer-3mr-q}
\begin{ruledtabular}
\begin{tabular}{cccccc}
Force constant & Value      &  & Force constant & Value      &    \\
%\begin{tabular}{ccc|ccc}
%Force constant & Value        & Unit  & Force constant & Value        & Unit     \\
\hline \Tstrut
?dimer-QFF-3MR-Q?
\end{tabular}
\end{ruledtabular}
\end{table}

\section[S2]{\label{sec:vibrations}Vibrational Frequencies}

\subsection{\label{sec:monomer-freq}CO$_2$ Monomer}

\begin{table}[H]
\caption{The vibrational energy levels for the CO$_2$ monomer obtained at the harmonic approximation and VSCF approximations with one, two and three mode representations (1MR, 2MR and 3MR). Values are shown in cm$^{-1}$.}
\begin{ruledtabular}
\begin{tabular}{lccccccc}
    & QFF &  Grid & QFF & Grid & QFF & Grid   \\  
  Mode & 1MR & 1MR & 2MR & 2MR & 3MR & 3MR   \\ 
\hline \Tstrut
?monomer-VSCF?
\end{tabular}
\end{ruledtabular}
\label{table:monomer-vscf}
\end{table}  

\begin{table}[h]
\caption{The vibrational energy levels for the CO$_2$ monomer obtained at the harmonic approximation and VMP2 approximations with one, two and three mode representations (1MR, 2MR and 3MR). Values are shown in cm$^{-1}$.}
\begin{ruledtabular}
\begin{tabular}{lccccccc}
    & QFF &  Grid & QFF & Grid & QFF & Grid   \\  
  Mode & 1MR & 1MR & 2MR & 2MR & 3MR & 3MR   \\ 
\hline \Tstrut
?monomer-VMP2?
\end{tabular}
\end{ruledtabular}
\label{table:monomer-vmp2}
\end{table}

\begin{table}[H]
    \caption{The vibrational configuration interaction energy levels for the CO$_2$ monomer obtained at the quartic force field (QFF) and Gauss-Hermite quadrature (Grid) approximations with one-, two- and three-mode representations (1MR, 2MR and 3MR). Values are shown in cm$^{-1}$.}
\begin{ruledtabular}
\begin{tabular}{lccccccc}
    & QFF &  Grid & QFF & Grid & QFF & Grid   \\  
  Mode & 1MR & 1MR & 2MR & 2MR & 3MR & 3MR   \\ 
\hline \Tstrut
?monomer-VCI?
\end{tabular}
\end{ruledtabular}
\label{table:monomer-vci}
\end{table}

\subsection{\label{sec:dimer-freq}CO$_2$ Dimer}

\begin{table}[H]
\caption{The vibrational energy levels for the CO$_2$ dimer obtained at the VSCF approximation with one, two and three mode representations (1MR, 2MR and 3MR). Values are shown in cm$^{-1}$.}
\begin{ruledtabular}
\begin{tabular}{lccccccc}
    & QFF &  Grid & QFF & Grid & QFF & Grid   \\  
  Mode & 1MR & 1MR & 2MR & 2MR & 3MR & 3MR   \\ 
\hline \Tstrut
?intra-VSCF?
\end{tabular}
\end{ruledtabular}
\label{table:intra-vscf}
\end{table}

\begin{table}[H]
\caption{The vibrational energy levels for the CO$_2$ dimer obtained at the VMP2 approximation with one, two and three mode representations (1MR, 2MR and 3MR). Values are shown in cm$^{-1}$.}
\begin{ruledtabular}
\begin{tabular}{lccccccc}
    & QFF &  Grid & QFF & Grid & QFF & Grid   \\  
  Mode & 1MR & 1MR & 2MR & 2MR & 3MR & 3MR   \\ 
\hline \Tstrut
?intra-VMP2?
\end{tabular}
\end{ruledtabular}
\label{table:intra-vmp2}
\end{table}


\begin{table}[H]
\caption{The vibrational energy levels for the CO$_2$ dimer obtained at the VCI approximation with one, two and three mode representations (1MR, 2MR and 3MR). Values are shown in cm$^{-1}$.}
\begin{ruledtabular}
\begin{tabular}{lccccccc}
    & QFF &  Grid & QFF & Grid & QFF & Grid   \\  
  Mode & 1MR & 1MR & 2MR & 2MR & 3MR & 3MR   \\ 
\hline \Tstrut
?intra-VCI-1?
\end{tabular}
\end{ruledtabular}
\label{table:intra-vci-1}
\end{table}


\begin{table}[b]
\caption{Continued table. The vibrational energy levels for the CO$_2$ dimer obtained at the VCI approximation with one, two and three mode representations (1MR, 2MR and 3MR). Values are shown in cm$^{-1}$.}
\begin{ruledtabular}
\begin{tabular}{lccccccc}
    & QFF &  Grid & QFF & Grid & QFF & Grid   \\  
  Mode & 1MR & 1MR & 2MR & 2MR & 3MR & 3MR   \\ 
\hline \Tstrut
?intra-VCI-2?
\end{tabular}
\end{ruledtabular}
\label{table:intra-vci-2}
\end{table}


\begin{table}[H]
\caption{The intermolecular vibrational energy levels for the CO$_2$ dimer obtained at the VSCF, VMP2, and VCI approximations. Values are shown in cm$^{-1}$.}
\begin{ruledtabular}
\begin{tabular}{lccc}
  Mode & VSCF & VMP2 & VCI    \\ 
\hline \Tstrut
?inter?
\end{tabular}
\end{ruledtabular}
\label{table:inter}
\end{table}

%\section{\label{sec:reproduce}S3 Accessibility and Reproducibility}
\section[S3]{\label{sec:reproduce}Accessibility and Reproducibility}


%\section*{\label{sec:level1}Supplementary Material}
%See supplementary material for the complete QFF force constants, VSCF, VMP2 and VCI vibrational frequencies and CI coefficients. 


%\input acknowledgement.tex   % input acknowledgement

\bibliographystyle{apsrev4-1}

\bibliography{anharmonic_v8}

%\bibliography{aipsamp}
\end{document}
%
% ****** End of file template.aps ******
